%        File: 03140299.tex
%     Created: Fri Jan 02 03:00 PM 2015 J
% Last Change: Fri Jan 02 03:00 PM 2015 J
%
\documentclass[10pt]{jarticle}

%%%%%%%%%%%%%%%%%%%%%
% to input Japanese %
%%%%%%%%%%%%%%%%%%%%%
\usepackage[japanese]{babel}

%%%%%%%%%%%%%%%%%%%%%
% to insert itembox %
%%%%%%%%%%%%%%%%%%%%%
\usepackage{ascmac}

%%%%%%%%%%%%%%%%%%%%%%%%%%
% to be standard a4paper %
%%%%%%%%%%%%%%%%%%%%%%%%%%
\usepackage{geometry}
\geometry{
  a4paper,
  total={210mm,297mm},
  left=20mm,
  right=20mm,
  top=20mm,
  bottom=40mm,
}

%%%%%%%%%%%%%%%%%%%%%
% to insert figures %
%%%%%%%%%%%%%%%%%%%%%
\usepackage[dvipdfmx]{graphicx}

%%%%%%%%%%%%%%%%%%%%%%%%%%
% to insert source codes %
%%%%%%%%%%%%%%%%%%%%%%%%%%
% \usepackage{listings, jlisting}
% \renewcommand{\lstlistingname}{list}
% \lstset{language=C,
%   basicstyle=\ttfamily\scriptsize,
%   commentstyle=\textit,
%   classoffset=1,
%   keywordstyle=\bfseries,
%   frame=tRBl,
%   framesep=5pt,
%   showstringspaces=false,
%   numbers=left,
%   stepnumber=1,
%   numberstyle=\tiny,
%   tabsize=2
% }

%%%%%%%%%%%%%%%%%%
% title & author %
%%%%%%%%%%%%%%%%%%
\title{ロボットインテリジェンス レポート課題A \\
      「ニューラルネット学習シミュレーション」}
\author{03-140299 東京大学機械情報工学科3年 和田健太郎}

%%%%%%%%%%%%%%%%%%
% begin document %
%%%%%%%%%%%%%%%%%%
\begin{document}
\maketitle

%%%%%%%%%%%%%%%%%%%%%%%%%%%%%%%%%%%%%%%%%%%%%%%%%%%%%%%%
\section{はじめに}
レポート課題として課題Aを選択し, 3層フィードフォワード型の
ニューラルネットとバックプロパゲーション学習をシミュレーション
するプログラムを作成し, 識別実験を行った. 
実験に利用したデータ群は
The MNIST database of handwritten digits
であり, このデータは過去に様々な分類器において
識別能力を図るために利用されている. \cite{mnist}

また, ノイズを加えた場合の性能変化, ノイズ耐性, 
中間ニューロンの役割, オートエンコーダを利用した画像特徴抽出による
識別性能変化について考察した. 
%%%%%%%%%%%%%%%%%%%%%%%%%%%%%%%%%%%%%%%%%%%%%%%%%%%%%%%%

%%%%%%%%%%%%%%%%%%%%%%%%%%%%%%%%%%%%%%%%%%%%%%%%%%%%%%%%
\section{ニューラルネット学習シミュレーション}
実験に利用したMNISTデータセットは, 28x28のグレースケールの手書き数字
画像データである. 画像に対する前処理はなしで
%%%%%%%%%%%%%%%%%%%%%%%%%%%%%%%%%%%%%%%%%%%%%%%%%%%%%%%%

%%%%%%%%%%%%%%%%%%%%%%%%%%%%%%%%%%%%%%%%%%%%%%%%%%%%%%%%
\section{ノイズによる性能変化}
%%%%%%%%%%%%%%%%%%%%%%%%%%%%%%%%%%%%%%%%%%%%%%%%%%%%%%%%

%%%%%%%%%%%%%%%%%%%%%%%%%%%%%%%%%%%%%%%%%%%%%%%%%%%%%%%%
\section{データ数とノイズ耐性}
%%%%%%%%%%%%%%%%%%%%%%%%%%%%%%%%%%%%%%%%%%%%%%%%%%%%%%%%

%%%%%%%%%%%%%%%%%%%%%%%%%%%%%%%%%%%%%%%%%%%%%%%%%%%%%%%%
%%%%%%%%%%%%%%%%%%%%%%%%%%%%%%%%%%%%%%%%%%%%%%%%%%%%%%%%
%%%%%%%%%%%%%%%%%%%%%%%%%%%%%%%%%%%%%%%%%%%%%%%%%%%%%%%%
%%%%%%%%%%%%%%%%%%%%%%%%%%%%%%%%%%%%%%%%%%%%%%%%%%%%%%%%
%%%%%%%%%%%%%%%%%%%%%%%%%%%%%%%%%%%%%%%%%%%%%%%%%%%%%%%%


%%%%%%%%%%%%%%%%%%%%%%%%%%
% to insert bibliography %
%%%%%%%%%%%%%%%%%%%%%%%%%%
\begin{thebibliography}{9}
%   \bibitem{inv1} Samuel R.Buss,"Introduction to Inverse Kinematics with Jacobian Transpose,Pseudoinverse and Damped Least Squares methods"
  \bibitem{mnist} http://yann.lecun.com/exdb/mnist/
\end{thebibliography}

%%%%%%%%%%%%%%%%
% end document %
%%%%%%%%%%%%%%%%
\end{document}